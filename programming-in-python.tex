\documentclass[a4paper, 10pt]{book}
%1m = 39.4 inch
%大18开 (18.5cm * 23cm)
%\usepackage[left=3.25cm, right=3.25cm, top=2.3cm,bottom=1.4cm]{geometry}
\usepackage{geometry}
\geometry{left=3.75cm,right=3.25cm,top=3cm,bottom=2.5cm}

%% en_preamble包含基本的宏包配置
\input{style/en_preamble}

%% 如果不写中文的话就不需要引用xecjk_preamble里面的配置
\input{style/xecjk_preamble}

%%%% 设置listings宏包用来粘贴源代码
%% 方便粘贴源代码,部分代码高亮功能
\usepackage{listings}
\usepackage{color}

\DeclareCaptionFont{red}{\color{red}}

%% 所要粘贴代码的编程语言
\lstloadlanguages{{[LaTeX]TeX}, {[ISO]C++}, {Java}, {Ruby}, {Python}, {Scala}}

% Some useful colors...
\definecolor{darkviolet}{rgb}{0.5,0,0.4}
\definecolor{darkgreen}{rgb}{0,0.4,0.2} 
\definecolor{darkblue}{rgb}{0.1,0.1,0.9}
\definecolor{darkgrey}{rgb}{0.5,0.5,0.5}
\definecolor{lightblue}{rgb}{0.4,0.4,1}

\definecolor{stringColor}{rgb}{0.16,0.00,1.00}
\definecolor{annotationColor}{rgb}{0.39,0.39,0.39}
\definecolor{keywordColor}{rgb}{0.50,0.00,0.33}
\definecolor{commentColor}{rgb}{0.25,0.50,0.37}
\definecolor{javadocColor}{rgb}{0.25,0.37,0.75}
\definecolor{jTagColor}{rgb}{0.50,0.62,0.75}
\definecolor{eTagColor}{rgb}{0.50,0.62,0.75}
\definecolor{lineNumberColor}{rgb}{0.47,0.47,0.47}

% Example:
% \lstdefinestyle{SQL}{
%     language={SQL},basicstyle=\ttfamily, 
%     moredelim=**[is][\btHL]{`}{`},
%     moredelim=**[is][{\btHL[fill=orange!30,draw=red,dashed,thin]}]{@}{@},
% }
% 
% A listing with {\btHL highlighting of all \textbf{important} elements} looks as follows:
% 
% \begin{lstlisting}[style=SQL]
% SELECT name, password `FROM` users @WHERE@ name=@UNION SELECT@
% \end{lstlisting}
%
\makeatletter
\newenvironment{btHighlight}[1][]
{\begingroup\tikzset{bt@Highlight@par/.style={#1}}\begin{lrbox}{\@tempboxa}}
{\end{lrbox}\bt@HL@box[bt@Highlight@par]{\@tempboxa}\endgroup}

\newcommand\btHL[1][]{%
  \begin{btHighlight}[#1]\bgroup\aftergroup\bt@HL@endenv%
}
\def\bt@HL@endenv{%
  \end{btHighlight}%   
  \egroup
}
\newcommand{\bt@HL@box}[2][]{%
  \tikz[#1]{%
    \pgfpathrectangle{\pgfpoint{1pt}{0pt}}{\pgfpoint{\wd #2}{\ht #2}}%
    \pgfusepath{use as bounding box}%
    \node[anchor=base west, fill=green!30,outer sep=0pt,inner xsep=0pt, inner ysep=0pt, rounded corners=0.5pt, minimum height=\ht\strutbox+1.1pt,#1]{\raisebox{1.1pt}{\strut}\strut\usebox{#2}};
  }%
}
\makeatother

%% 设置listings宏包的一些全局样式
%% 参考http://hi.baidu.com/shawpinlee/blog/item/9ec431cbae28e41cbe09e6e4.html
\lstset{
numberbychapter=true,
breakatwhitespace=true,
showstringspaces=false,              %% 设定是否显示代码之间的空格符号
basicstyle=\scriptsize\ttfamily,           %% 设定字体大小\tiny, \scriptsize, \footnotesize, \small, \Large等等
keywordstyle=\color{keywordColor}\bfseries,
commentstyle=\color{red!50!green!50!blue!50},      
stringstyle=\color{stringColor},  
escapechar=`,                        %% 中文逃逸字符,用于中英混排
xleftmargin=1.5em,xrightmargin=0em, aboveskip=1.0em,
breaklines,                          %% 这条命令可以让LaTeX自动将长的代码行换行排版
extendedchars=false,                 %% 这一条命令可以解决代码跨页时,章节标题,页眉等汉字不显示的问题
frameround=fttt,
captionpos=top,
belowcaptionskip=1em
}

\lstdefinestyle{numbers}{
   numbers=left,
   numberstyle=\tiny\color{lineNumberColor}\lstfontfamily,
   stepnumber=1,
   numbersep=1em,
}

\lstdefinestyle{C++}{
   language=C++,
   texcl=true,
   prebreak=\textbackslash,
   breakindent=1em,
   emphstyle=\bfseries,
   keywordstyle=\color{keywordColor}\bfseries, %% 关键字高亮
   stringstyle=\color{stringColor},  
   morekeywords={alignas, alignof, char16_t, char32_t, constexpr, decltype, noexcept, nullptr, static_assert, thread_local, override, final},
   moredelim=**[is][\btHL]{@}{@},
   %style=numbers,
   %frame=leftline,                     %% 给代码加框
   %framerule=2pt,
   %rulesep=5pt
}

\lstnewenvironment{minicpp}[1][]
  {\setstretch{1}
  \lstset{style=C++, xleftmargin=0.0em, #1}}
  {}

\lstnewenvironment{c++}[1][]
  {\setstretch{1}
  \lstset{style=C++, #1}}
  {}


%\captionsetup[lstlisting]{textfont=red}
%{labelfont=bf, singlelinecheck=off, labelsep=space, textfont=red}

\lstdefinestyle{Java}{
   language=Java,
   texcl=true,
   prebreak=\textbackslash,
   breakindent=1em,
   keywordstyle=\bfseries, %% 关键字高亮
   morekeywords={}
   style=numbers,
   %frame=leftline,                     %% 给代码加框
   %framerule=2pt,
   %rulesep=5pt
}

\lstnewenvironment{java}[1][]
  {\setstretch{1}
  \lstset{style=Java, #1}}
  {}

\lstdefinestyle{Ruby}{
   language=Java,
   texcl=true,
   prebreak=\textbackslash,
   breakindent=1em,
   keywordstyle=\bfseries, %% 关键字高亮
   morekeywords={}
   style=numbers,
   %frame=leftline,                     %% 给代码加框
   %framerule=2pt,
   %rulesep=5pt
}

\lstnewenvironment{ruby}[1][]
  {\setstretch{1}
  \lstset{style=Ruby, #1}}
  {}

\lstdefinestyle{Python}{
   language=Python,
   texcl=true,
   prebreak=\textbackslash,
   breakindent=1em,
   emphstyle=\bfseries,
   keywordstyle=\color{keywordColor}\bfseries, %% 关键字高亮
   stringstyle=\color{stringColor},  
   morekeywords={with, as},
   moredelim=**[is][\btHL]{@}{@},
   %frame=leftline,                     %% 给代码加框
   %framerule=2pt,
   %rulesep=5pt
}

\lstnewenvironment{python}[1][]
  {\setstretch{1}
  \lstset{style=Python, #1}}
  {}

\lstdefinestyle{Scala}{
   language=Scala,
   texcl=true,
   prebreak=\textbackslash,
   breakindent=1em,
   keywordstyle=\bfseries, %% 关键字高亮
   morekeywords={}
   style=numbers,
   %frame=leftline,                     %% 给代码加框
   %framerule=2pt,
   %rulesep=5pt
}

\lstnewenvironment{scala}[1][]
  {\setstretch{1}
  \lstset{style=Scala, #1}}
  {}  

\renewcommand{\lstlistingname}{示例代码}
\renewcommand\thefigure{\thechapter-\arabic{figure}}

\newcommand\refcode[1]{{\itshape \lstlistingname\ascii{\ref{code:#1}(第\pageref{code:#1}页)}}}



% \usepackage[
%   placement=center,
%   angle=45,
%   scale=20,
%   color=black!40,
%   %hshift=60,
%   %vshift=-5
% ]{background}

% \backgroundsetup{contents={样章}}
% \backgroundsetup{contents={\includegraphics[width=0.2\textwidth]{figures/cock.jpg}}}

\newcommand{\myclearpage}{\clearpage{\pagestyle{empty}\cleardoublepage}}
\newcommand{\mydedicate}{\clearpage{\pagestyle{empty}\cleardedicatepage}}

\begin{document}

\frontmatter
\pagestyle{empty}

\def\titlename{Python编程}
\def\subtitle{Programming in Python}
\def\authors{刘光聪\ 著}
% \def\orgnization{\ascii{}}

\input{style/title}
\myclearpage

\mydedicate

\tableofcontents
\myclearpage

\def\thelstlisting{\thechapter-\arabic{lstlisting}}
%% 中文习惯是设定首行缩进为2em。
%% 注意此设置一定要在document环境之中,这可能与\setlength作用范围相关
\setlength{\parindent}{2em}

%%%%%%%%%%%%%%%%%%%%%%
%%开始正文,页面计数从正文开始
\mainmatter
\setcounter{page}{1}
\pagestyle{fancy}

\begin{savequote}[45mm]
\ascii{Any fool can write code that a computer can understand. Good programmers write code that humans can understand.}
\qauthor{\ascii{- Martin Flower}}
\end{savequote}

\chapter{魔术方法} 
\label{ch:magic-method}

\section{迭代器}

\begin{content}

\subsection{迭代协议}

\begin{nodiff}{迭代器}
 \begin{python}
import tensorflow as tf

hello = tf.contant("hello, world")

with tf.Session() as sess:
  sess.run(hello)
 \end{python}
\end{nodiff}

\end{content}


%%%%%%%%%%%%%%%%%%%%%
\appendix

%\part{附录}
%\input{contents/appendix/k8s-cluster.tex}

%%%%%%%%%%%%%%%%%%%%
\backmatter
%\listoffigures
%\myclearpage

%\listoftables
%\myclearpage

\bibliographystyle{alpha}
\renewcommand\bibname{参考文献}
\begin{thebibliography}{20}

\ascii{

\bibitem{agile-principles-1st} Robert C. Martin.
  \newblock \emph{Agile Software Development, Principles, Patterns, and Practices}.

\bibitem{software-craftsman} Robert C. Martin.
  \newblock \emph{The Software Craftsman: Professionalism, Pragmatism, Pride}. 

\bibitem{clean-architecture} Robert C. Martin.
  \newblock \emph{Clean Architecture: A Craftsman's Guide to Software Structure and Design}.  

\bibitem{clean-code} Robert C. Martin. 
  \newblock \emph{Clean Code: A Handbook of Agile Software Craftsmanship}. 

\bibitem{clean-coder} Robert C. Martin.
  \newblock \emph{The Clean Coder: A Code of Conduct for Professional Programmers}. 

\bibitem{tdd-by-example} Kent Beck.
  \newblock \emph{Test Driven Development: By Example}.

\bibitem{xp-explained} Kent Beck.
  \newblock \emph{Extreme Programming Explained: Embrace Change}.

\bibitem{impl-patterns} Kent Beck.
  \newblock \emph{Implementation Patterns}.

\bibitem{refactoring} Martin Fowler, Kent Beck, John Brant, William Opdyke, Don Roberts. 
  \newblock \emph{Refactoring: Improving the Design of Existing Code}. 

\bibitem{dsl} Martin Fowler. 
  \newblock \emph{Domain-Specific Languages}. 

\bibitem{patters-arch} Martin Fowler. 
  \newblock \emph{Patterns of Enterprise Application Architecture}. 

\bibitem{microservices} Sam Newman. 
  \newblock \emph{Building Microservices: Designing Fine-Grained Systems}. 

\bibitem{gof} Erich Gamma, Richard Helm, Ralph Johnson, John Vlissides.
  \newblock \emph{Design Patterns: Elements of Reusable Object-Oriented Software}.

\bibitem{pragmatic-programmer} Andrew Hunt, David Thomas.
  \newblock \emph{The Pragmatic Programmer: From Journeyman to Master}.  

\bibitem{code-complete} Steve McConnell.
  \newblock \emph{Code Complete: A Practical Handbook of Software Construction, 2th Edition}.  

\bibitem{working-with-legacy-code} Michael Feathers.
  \newblock \emph{Working Effectively with Legacy Code}.  

\bibitem{emc} Scott Meyers.
  \newblock \emph{Effective Modern C++: 42 Specific Ways to Improve Your Use of C++11 and C++14}.   

\bibitem{eff-cpp} Scott Meyers.
  \newblock \emph{Effective C++: 55 Specific Ways to Improve Your Programs and Designs}.   

\bibitem{more-eff-cpp} Scott Meyers.
  \newblock \emph{More Effective C++: 35 New Ways to Improve Your Programs and Designs}.   

\bibitem{eff-stl} Scott Meyers.
  \newblock \emph{Effective STL: 50 Specific Ways to Improve Your Use of the Standard Template Library}.   

\bibitem{eff-java} Joshua Bloch.
  \newblock \emph{Effective Java, 3rd Edition}.   

\bibitem{prog-scala} Martin Odersky, Lex Spoon, Bill Venners.
  \newblock \emph{Programming in Scala, 3rd Edition}.   

\bibitem{prog-scala} Martin Odersky, Lex Spoon, Bill Venners.
  \newblock \emph{Programming in Scala, 3rd Edition}.   

\bibitem{tdd-c} James W. Grenning.
  \newblock \emph{Test Driven Development for Embedded C}.   


\bibitem{cpp-template} David Vandevoorde, Nicolai M. Josuttis, Douglas Gregor.
  \newblock \emph{C++ Templates: The Complete Guide}.   

\bibitem{cpp-primer} Stanley B. Lippman, Josée Lajoie, Barbara E. Moo
  \newblock \emph{C++ Primer, 5th Edition}.

\bibitem{programming-bs} Bjarne Stroustrup.
  \newblock \emph{Programming: Principles and Practice Using C++, 2th Edition}.

\bibitem{programming-cpp} Bjarne Stroustrup.
  \newblock \emph{The C++ Programming Language, 4th Edition}.

\bibitem{cpp-concurrency} Anthony Williams.
  \newblock \emph{C++ Concurrency in Action: Practical Multithreading}.

\bibitem{large-cpp-design} John Lakos.
  \newblock \emph{Large-Scale C++ Software Design}.

\bibitem{cpp-coding-std} Herb Sutter, Andrei Alexandrescu.
  \newblock \emph{C++ Coding Standards: 101 Rules, Guidelines, and Best Practices}.

\bibitem{modern-cpp-design} Andrei Alexandrescu.
  \newblock \emph{Modern C++ Design: Generic Programming and Design Patterns Applied}.

\bibitem{ddd} Eric Evans.
  \newblock \emph{Domain-Driven Design: Tackling Complexity in the Heart of Software}.

\bibitem{ddd-prac} Scott Millett, Nick Tune.
  \newblock \emph{Patterns, Principles, and Practices of Domain-Driven Design}.

\bibitem{story-mapping} Jeff Patton, Peter Economy.
  \newblock \emph{User Story Mapping: Discover the Whole Story, Build the Right Product}.


\bibitem{man-month} Frederick P. Brooks Jr.
  \newblock \emph{The Mythical Man-Month: Essays on Software Engineering}.

\bibitem{sicp} Harold Abelson, Gerald Jay Sussman.
  \newblock \emph{Structure and Interpretation of Computer Programs, 2nd Edition}.

\bibitem{c-prog} Brian W. Kernighan, Dennis M. Ritchie.
  \newblock \emph{The C Programming Language, 2nd Edition}.

\bibitem{go-prog} Alan A. A. Donovan, Brian W. Kernighan.
  \newblock \emph{The Go Programming Language}.

\bibitem{pro-git} Scott Chacon, Ben Straub.
  \newblock \emph{Pro Git, 2nd Edition}.

\bibitem{cod} David A. Patterson, John L. Hennessy.
  \newblock \emph{Computer Organization and Design: The Hardware/Software Interface, 5th Edition}.

\bibitem{intro-algo} Thomas H. Cormen, Charles E. Leiserson, Ronald L. Rivest, Clifford Stein.
  \newblock \emph{Introduction to Algorithms, 3rd Edition}.

}

\end{thebibliography}

\endinput

\end{document}
