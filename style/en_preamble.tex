%%%%%%%%------------------------------------------------------------------------
%%%% 日常所用宏包

%%设置行间距
\usepackage{setspace}

%% 控制项目列表
\usepackage{enumerate}

%% 多栏显示
\usepackage{multicol}

\usepackage[most]{tcolorbox}

%% hyperref宏包,生成可定位点击的超链接,并且会生成pdf书签
\usepackage[%
    pdfstartview=FitH,%
    CJKbookmarks=true,%
    bookmarks=true,%
    bookmarksnumbered=true,%
    bookmarksopen=true,%
    colorlinks=true,%
    citecolor=blue,%
    linkcolor=blue,%
    anchorcolor=green,%
    urlcolor=blue%
]{hyperref}

%% 控制标题
\usepackage{titlesec}

%% 控制表格样式
\usepackage{booktabs}

%% 控制目录
\usepackage{titletoc}

%% 控制字体大小
\usepackage{type1cm}

%% 首行缩进,用\noindent取消某段缩进
\usepackage{indentfirst}

%% 支持彩色文本、底色、文本框等
\usepackage{color,xcolor}

%% AMS LaTeX宏包
\usepackage{amsmath}
\usepackage{amssymb}

\usepackage{multirow}

%% 一些特殊符号
% \usepackage{bbding}

%% 支持引用
% \usepackage{cite}

%% LaTeX一些特殊符号宏包
% \usepackage{latexsym}

%% 数学公式中的黑斜体
% \usepackage{bm}

%% 调整公式字体大小:\mathsmaller, \mathlarger
% \usepackage{relsize}

%% 生成索引
% \makeindex

%%%% 基本插图方法
%% 图形宏包
\usepackage{graphicx}
\usepackage{float}
%% 如果插入的图片没有指定扩展名,那么依次搜索下面的扩展名所对应的文件
\DeclareGraphicsExtensions{.pdf,.eps,.png,.jpg}
%% 让 latex 从 .bb 中读取 Bounding Box 信息
%\DeclareGraphicsRule{.jpg}{eps}{.bb}{}
%\DeclareGraphicsRule{.png}{eps}{.bb}{}
%\DeclareGraphicsRule{.pdf}{eps}{.bb}{}

%% 多个图形并排,参加lnotes.pdf
%\usepackage{subfig}
\usepackage{subfigure}

\usepackage{caption}
\captionsetup{font={sf, scriptsize}, labelfont={bf}, skip=15pt}
\DeclareCaptionLabelSeparator{colon}{~~}

\usepackage[perpage,stable]{footmisc}

\usepackage{longtable}
% \begin{figure}[htbp]               %% 控制插图位置
%   \setlength{\abovecaptionskip}{0pt}
%   \setlength{\belowcaptionskip}{10pt}
                                     %% 控制图形和上下文的距离
%   \centering                       %% 使图形居中显示
%   \includegraphics[width=0.8\textwidth]{CTeXLive2008.jpg}
                                     %% 控制图形显示宽度为0.8\textwidth
%   \caption{CTeXLive2008安装过程} \label{fig:CTeXLive2008}
                                     %% 图形题目和交叉引用标签
% \end{figure}
%%%% 基本插图方法结束

%%%% pgf/tikz绘图宏包设置
\usepackage{pgf,tikz}
\usetikzlibrary{shapes,automata,snakes,backgrounds,arrows}
\usetikzlibrary{mindmap, trees,  calendar}
\usetikzlibrary{positioning}
\usepackage{pgf-umlsd}
%% 可以直接在latex文档中使用graphviz/dot语言,
%% 也可以用dot2tex工具将dot文件转换成tex文件再include进来
%% \usepackage[shell,pgf,outputdir={docgraphs/}]{dot2texi}
%%%% pgf/tikz设置结束


%%%% fancyhdr设置页眉页脚
%% 页眉页脚宏包
\usepackage{fancyhdr}

%% 页眉页脚风格
\pagestyle{fancy}

%%这两行代码是修改\leftmark和\rightmark的经典模式
\renewcommand{\chaptermark}[1]{\markboth{{\hei {第\thechapter{}章}}\hspace 1  #1}{}}
\renewcommand{\sectionmark}[1]{\markright{\thesection{} #1}}

%% 清空当前页眉页脚的默认设置
\fancyhf{}

\renewcommand{\headrulewidth}{0.4pt}
\renewcommand{\footrulewidth}{0.4pt}

%第{\couriernew\thechapter{}}章
%%下面开始修改页眉和页脚
\fancyhead[RE]{\fangsong \leftmark}
\fancyhead[LO]{\fangsong \rightmark}
\fancyhead[RO, LE]{\small \thepage}
\fancypagestyle{plain}{%
  \fancyhead{} % get rid of headers
  \renewcommand{\headrulewidth}{0pt} % and the line.
}

%%定义空白页面
\makeatletter
\def\cleardoublepage{\clearpage\if@twoside \ifodd\c@page\else
  \hbox{}
  \vspace*{\fill}
  \begin{center}
   {\sffamily\large}
   \end{center}
   \vspace{\fill}
   \thispagestyle{empty}
   \newpage
   \if@twocolumn\hbox{}\newpage\fi\fi\fi}
\makeatother

\makeatletter
\def\cleardedicatepage{\clearpage
  \hbox{}
  \vspace*{\fill}
  \begin{center}
   {\sffamily\Large 献给我的女儿刘楚溪}
   \end{center}
   \vspace{\fill}
   \thispagestyle{empty}
   \newpage
   \if@twocolumn\hbox{}\newpage\fi}
\makeatother

%% 有时会出现\headheight too small的warning
\setlength{\headheight}{15pt}

%%%% fancyhdr设置结束

%%设置行间距离
\usepackage{framed}  
%%%% listings宏包设置结束

%%%% 附录设置
\usepackage[title,titletoc,header]{appendix}
%%%% 附录设置结束

%%%% 日常宏包设置结束
%%%%%%%%------------------------------------------------------------------------

%%%%%%%%------------------------------------------------------------------------
%%%% 英文字体设置结束
%% 这里可以加入自己的英文字体设置
%%%%%%%%------------------------------------------------------------------------

%%%%%%%%------------------------------------------------------------------------
%%%% 设置常用字体字号,与MS Word相对应

%% 一号, 1.4倍行距
\newcommand{\yihao}{\fontsize{26pt}{36pt}\selectfont}
%% 二号, 1.25倍行距
\newcommand{\erhao}{\fontsize{22pt}{28pt}\selectfont}
%% 小二, 单倍行距
\newcommand{\xiaoer}{\fontsize{18pt}{18pt}\selectfont}
%% 三号, 1.5倍行距
\newcommand{\sanhao}{\fontsize{16pt}{24pt}\selectfont}
%% 小三, 1.5倍行距
\newcommand{\xiaosan}{\fontsize{15pt}{22pt}\selectfont}
%% 四号, 1.5倍行距
\newcommand{\sihao}{\fontsize{14pt}{21pt}\selectfont}
%% 半四, 1.5倍行距
\newcommand{\bansi}{\fontsize{13pt}{19.5pt}\selectfont}
%% 小四, 1.5倍行距
\newcommand{\xiaosi}{\fontsize{12pt}{18pt}\selectfont}
%% 大五, 单倍行距
\newcommand{\dawu}{\fontsize{11pt}{11pt}\selectfont}
%% 五号, 单倍行距
\newcommand{\wuhao}{\fontsize{10.5pt}{10.5pt}\selectfont}
%%%%%%%%------------------------------------------------------------------------

%%%%%%%%------------------------------------------------------------------------
%%%% 一些个性设置

%% 设定页码方式,包括arabic、roman等方式
%% \pagenumbering{arabic}

%% 有时LaTeX无从断行,产生overfull的错误,这条命令降低LaTeX断行标准
%% \sloppy

%% 设定目录显示深度\tableofcontents
%% \setcounter{tocdepth}{2}
%% 设定\listoftables显示深度
%% \setcounter{lotdepth}{2}
%% 设定\listoffigures显示深度
%% \setcounter{lofdepth}{2}

%% 中文破折号,据说来自清华模板
\newcommand{\pozhehao}{\kern0.3ex\rule[0.8ex]{2em}{0.1ex}\kern0.3ex}

%% 设定itemize环境item的符号
\renewcommand{\labelitemi}{$\bullet$}

%\makeatletter
%\@addtoreset{lstlisting}{section} 
%\makeatother

\usepackage{array,tabularx}
\usepackage{colortbl}

\newenvironment{enum}
{
  \begin{spacing}{1.2}
  % \begin{enumerate}[1.]
  \begin{enumerate}
    \setlength{\itemsep}{0pt} 
    \setlength{\itemindent}{2em}
    %\setlength{\listparindent}{2em}
}{%
  \end{enumerate}
  \end{spacing}
}

\newenvironment{nitemize}
{
  \begin{itemize}
    \setlength{\itemsep}{0pt} 
    \setlength{\itemindent}{2em}
}{%
  \end{itemize}
}


\newcommand{\suggest}[1]{
\tikzstyle{mybox} = [draw=black, very thick,
rectangle, rounded corners, inner sep=9pt, inner ysep=20pt]
\tikzstyle{fancytitle} =[fill=white, text=black, ellipse]
\noindent
\begin{tikzpicture}
\node [mybox] (box){%
\begin{minipage}{\textwidth}
\fangsong
#1
\end{minipage}
};
\node[fancytitle, right=10pt] at (box.north west) {\emph{建议}};
% \node[fancytitle, rounded corners] at (box.east) {$\clubsuit$};
\end{tikzpicture}
}

\newcommand{\notice}[1]{
\tikzstyle{mybox} = [draw=black, very thick,
rectangle, rounded corners, inner sep=9pt, inner ysep=20pt]
\tikzstyle{fancytitle} =[fill=white, text=black]
\noindent
\begin{tikzpicture}
\node [mybox] (box){%
\begin{minipage}{\textwidth}
\fangsong
#1
\end{minipage}
};
\node[fancytitle, right=10pt] at (box.north west) {\emph{注意}};
%\node[fancytitle, rounded corners] at (box.east) {$\clubsuit$};
\end{tikzpicture}
}

\newcommand{\tip}[1]{
\tikzstyle{mybox} = [draw=black, very thick,
rectangle, rounded corners, inner sep=9pt, inner ysep=20pt]
\tikzstyle{fancytitle} =[fill=white, text=black]
\noindent
\begin{tikzpicture}
\node [mybox] (box){%
\begin{minipage}{\textwidth}
\fangsong
#1
\end{minipage}
};
\node[fancytitle, right=10pt] at (box.north west) {\emph{提示}};
%\node[fancytitle, rounded corners] at (box.east) {$\clubsuit$};
\end{tikzpicture}
}

\newcommand\refch[1]{\ascii{第\ref{ch:#1}章(\nameref{ch:#1})}}
\newcommand\refsec[1]{\ascii{\ref{sec:#1}节(\nameref{sec:#1})}}

% \newcommand\eitem[1]{\item {\optima {#1}}}
\newcommand\eitem[1]{\item {\itshape {#1}}}
\newcommand\cpp{\ascii{C\nobreak+\nobreak+}}
\newcommand\clang{\ascii{C}}
\newcommand\tf{\ascii{TensorFlow}}
\newcommand\docker{\ascii{Docker}}
\newcommand\kube{\ascii{Kubernetes}}
\newcommand\apiserver{\ascii{API Server}}
\newcommand\etcd{\ascii{ETCD}}
\newcommand\cm{\ascii{Controller Manager}}
\newcommand\sche{\ascii{Scheduler}}
\newcommand\Pod{\ascii{Pod}}
\newcommand\svc{\ascii{Service}}
\newcommand\ep{\ascii{Endpoint}}
\newcommand\vip{\ascii{ClusterIP}}

\newcommand\quo[1]{“#1”}

\newcommand\percent[1]{\ascii{#1\%}}

\newcommand{\trans}{\emph{事务}}
\newcommand{\act}{\emph{操作}}
\newcommand{\seqact}{\emph{顺序操作}}
\newcommand{\conact}{\emph{并发操作}}
\newcommand{\atomact}{\emph{基本操作}}
\newcommand{\syncact}{\emph{同步操作}}
\newcommand{\asynact}{\emph{异步操作}}
\newcommand{\action}[1]{\emph{\ascii{\itshape\_\_#1}}}
\newcommand{\sigwait}{\action{sig\_wait}}
\newcommand{\sigsync}{\action{sig\_sync}}
\newcommand{\sigreply}{\action{sig\_reply}}
\newcommand{\timerprot}{\action{timer\_prot}}
\newcommand{\unknownevet}{\ascii{UNKNOWN\_EVENT}}
\newcommand{\transdsl}{\ascii{Transaction DSL}}
\newcommand{\oper}[1]{\ascii{Action#1}}
\newcommand{\protproc}{\ascii{prot\_procedure}}

%\newcommand{\code}[1]{\ascii{\small{\texttt{#1}}}}
\newcommand{\code}[1]{\ascii{\footnotesize{\texttt{#1}}}}
\newcommand{\script}[1]{\ascii{\scriptsize{\texttt{#1}}}}


\newcommand{\inlinetitle}[1]{\large{\emph{#1}}}

%\newcommand{\Email}{\begingroup \def\UrlLeft{<}\def\UrlRight{>} \urlstyle{tt}\Url}
%\def\mailto|#1|{\href{mailto:#1}{Email|#1|}}
\newcommand{\contrib}[2]{#1\quad{\small\quad\textit{#2}}\\[1ex]}


\newcommand{\upcite}[1]{\textsuperscript{\cite{#1}}}

% todo list

% 使用方法
% \todobox{
%   \begin{todolist}
%   \item[\done] Frame the problem
%   \item Write solution
%   \item[\wontfix] profit
%   \end{todolist}
% }

\newcommand{\todobox}[1]{
\tikzstyle{mybox} = [draw=black, very thick,
rectangle, rounded corners, inner sep=9pt, inner ysep=20pt]
\tikzstyle{fancytitle} =[fill=white, text=black]
\noindent
\begin{tikzpicture}
\node [mybox] (box){%
\begin{minipage}{\textwidth}
\fangsong
#1
\end{minipage}
};
\node[fancytitle, right=10pt] at (box.north west) {\emph{TODO列表}};
%\node[fancytitle, rounded corners] at (box.east) {$\clubsuit$};
\end{tikzpicture}
}

\newtcolorbox{episode}[1]{enhanced jigsaw,breakable,pad at break*=1mm,colback=red!5!white, colframe=red!75!black,fonttitle=\bfseries, colbacktitle=red!85!black,enhanced, watermark color=white,watermark text=\Roman{tcbbreakpart},center title,title=#1}

\newtcolorbox[auto counter,number within=chapter]{nodiff}[1]{left=0mm,right=0mm,enhanced jigsaw,breakable,pad at break*=1mm,colback=red!5!white,colframe=red!75!black,fonttitle=\footnotesize\bfseries, colbacktitle=red!85!black,enhanced, attach boxed title to top left={yshift=-2mm}, watermark color=white,title=\code{#1}}

\newtcolorbox[auto counter,number within=chapter]{diff}[1]{toggle enlargement=forced,spread inwards=-2cm,spread outwards=-1.5cm,left=0mm,right=0mm,enhanced jigsaw,breakable,pad at break*=1mm,colback=red!5!white,colframe=red!75!black,fonttitle=\footnotesize\bfseries, colbacktitle=red!85!black,enhanced, attach boxed title to top left={yshift=-2mm}, watermark color=white,title=\code{#1},sidebyside,sidebyside align=top,sidebyside gap=2mm}

\newtcolorbox{inlinediff}{opacityframe=0.0,opacityback=0.0,boxrule=0.0mm,left=0mm,right=0mm,sidebyside,sidebyside align=top,sidebyside gap=2mm}

\newtcolorbox[auto counter,number within=chapter]{topdiff}[1]{toggle enlargement=forced,spread inwards=-2cm,spread outwards=-1.5cm,bottomrule=0mm,left=0mm,right=0mm,enhanced jigsaw,breakable,pad at break*=1mm,colback=red!5!white,colframe=red!75!black,fonttitle=\bfseries, colbacktitle=red!85!black,enhanced, attach boxed title to top left={yshift=-2mm}, watermark color=white,title=\code{#1},sidebyside,sidebyside align=top,sidebyside gap=2mm}

\newtcolorbox{mindiff}{toggle enlargement=forced,spread inwards=-2cm,spread outwards=-1.5cm, toprule=0mm,bottomrule=0mm,left=0mm,right=0mm,enhanced jigsaw,breakable,pad at break*=1mm,colback=red!5!white,colframe=red!75!black, colbacktitle=red!85!black,enhanced, attach boxed title to top left={yshift=-2mm}, watermark color=white,sidebyside,sidebyside align=top,sidebyside gap=2mm}

\newtcolorbox{lastdiff}{toggle enlargement=forced,spread inwards=-2cm,spread outwards=-1.5cm,toprule=0mm,left=0mm,right=0mm,enhanced jigsaw,breakable,pad at break*=1mm,colback=red!5!white,colframe=red!75!black, colbacktitle=red!85!black,enhanced, attach boxed title to top left={yshift=-2mm}, watermark color=white,sidebyside,sidebyside align=top,sidebyside gap=2mm}

\newtcolorbox{colortable}[2]{leftright skip=2cm,enhanced,fonttitle=\bfseries,fontupper=\footnotesize\sffamily,
colback=red!5!white,colframe=red!75!black, colbacktitle=red!85!black,center title,tabularx*={\arrayrulewidth0.25mm}{#1},title=#2}

\usepackage{enumitem,amssymb}
\usepackage{pifont}

\newlist{todolist}{itemize}{2}
\setlist[todolist]{label=$\square$}

\newcommand{\cmark}{\ding{51}}%
\newcommand{\xmark}{\ding{55}}%
\newcommand{\done}{\rlap{$\square$}{\raisebox{2pt}{\large\hspace{1pt}\cmark}}%
\hspace{-2.5pt}}
\newcommand{\wontfix}{\rlap{$\square$}{\large\hspace{1pt}\xmark}}


%% 设定正文字体大小
% \renewcommand{\normalsize}{\sihao}

%%%% 个性设置结束
%%%%%%%%------------------------------------------------------------------------




%%%%%%%%------------------------------------------------------------------------
%%%% bibtex设置

%% 设定参考文献显示风格

%%%% bibtex设置结束
%%%%%%%%------------------------------------------------------------------------
